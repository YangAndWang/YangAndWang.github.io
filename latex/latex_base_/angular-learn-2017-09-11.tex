\documentclass{article}

% 中文
\usepackage[UTF8]{ctex}
% 自动缩进
\usepackage{indentfirst}
% 支持跳转,标题的超链接
%\usepackage{hyperref}
\usepackage[colorlinks=true,linkcolor=blue]{hyperref}


% begin end环境
% flushleft:环境中的内容居左。
% flushright:环境中的内容居右。
% center:环境中的内容居中。
% itemize:无编号列表
% enumerate:有编号列表
% description:带描述列表
% quote:引用,使得整段缩进
% verse:诗歌专用,\\可以断行,两个空行的分段则生成一个空行。
% begin end 环境

\title{angular学习}
\author{汪洋}


\begin{document}
\maketitle

\newpage
\tableofcontents
\newpage
 
\section{Pipe 管道}
只有当string,number值改变,或者Object的引用发生改变,才会触发管道


@Pipe(\{
  name: 'pipeName',
  pure: false
\})

将管道声明为不纯的管道,不纯的管道可以监听对象属性的改变,但是速度会更慢.

\section{Animation 动画}
\subsection{configure}
@Component(\{ animations:[] \})
\par 组件注解中添加额外的属性 animations
\{
\par\ \ trigger('animationName',[
\par\ \ \ \ state('state1', style(\{\})),
\par\ \ \ \ state('state2', style(\{\})),
\par\ \ \ \ transition('state1 =$>$ state2', animation('100ms ease-out')) 
\par\ \ ])
\}

transition 有三个方向 =$>$, $<$= 和 $<$=$>$

state有三种 void, * 和 自定义
\begin{enumerate}
\item void 表示当刚开始挂载到Dom节点或者从Dom移除时
\item * 表示任意state
\item state 自定义的state
\end{enumerate}

void =$>$ * 的别名是 :enter
\par * =$>$ void 的别名是 :leave

如果多个状态之间的变化是相同的时间和特效.可以将状态变化写在一起,使用,分开.

状态变化的样式也可以内联到transition中\\
transition('inactive =$>$ active', $[$
\par\ \ style(\{
\par\ \ \ \ backgroundColor: '\#cfd8dc',
\par\ \ \ \ transform: 'scale(1.3)'
\par\ \ \}),
\par\ \ animate('80ms ease-in', style(\{
\par\ \ \ \ backgroundColor: '\#eee',
\par\ \ \ \ transform: 'scale(1)'
\par\ \ \}))
$]$)

style:中的属性单位有 px, em 和\%,当只提供数字而不提供单位时,angular会使用默认的单位px,当不知道具体的数字时,可以使用*号代替,angular会自动计算.

timing delay easing\ ,
timing easing\ 是animation的属性,即运行时间,延迟时间以及动画特效.

\subsection{keyFrame}
\par animate(300,keyframes$[$
\par\ \ style(\{...styles,offset:0\}),
\par\ \ style(\{...styles,offset:1\}) 
\par $]$);

offset 是动画的时间 是一个小数(相对整体时间的百分比).这个属性是可选的,当没有时,会按照keyframes的数量来平均分配时间.
\subsection{callbacks,start,done}

(@animationName.start)

(@animationName.done)

[@animationName]="'initState'"

\section{Forms}
\subsection{user input}
\subsubsection{Input}
(keyup)="onKey(\$event)"

onKey(event: any)\{

\}

onKey(event: KeyboardEvent)\{
\par\ \ (($<$HTMLInputElement$>$)event.target) 
\par\}


$<$input \#name /$>$
\par $<$p$>$\{\{name.value\}\}$<$/p$>$


$<$input \#name (keyup)="onKey(name.value)" /$>$
\par $<$p$>$\{\{nameValue\}\}$<$/p$>$



$<$input \#nameInput (keyup.enter)="onKey(nameInput.value)" /$>$
\par $<$p$>$\{\{name\}\}$<$/p$>$
%\textbackslash\textbackslash 
\par//keyup.enter 是对键盘Enter键按下,按起之后

$<$input \#nameInput (keyup.enter)="update(nameInput.value)" (blur)="update(nameInput.value)" /$>$
\par $<$p$>$\{\{name\}\}$<$/p$>$

\subsection{form,template-driven,reactive-form}

FormControl 有 value, status, pristine, untouched
status 是 VALID, INVALID, PENDING, or DISABLED值中的一个
pristine 是和 dirty 取反

\subsubsection{template-driven}
\begin{enumerate}
\item 需要引用 FormModule
\item 输入双向绑定 [(ngModel)] 
\item name 属性,angular Form需要一个name来注册
\item angular 
\end{enumerate}

将自动添加3种类(即 className) ng-(un)touched, ng-(dirty,pristine), ng-(in)valid

内建的几个验证类
\begin{enumerate}
\item required
\item minlength="4"
% \item 
\end{enumerate}

自定义验证方法
\begin{enumerate}
\item 工厂方法
\par export function forbiddenNameValidator(nameRe: RegExp): ValidatorFn \{
\par\ \ return (control: AbstractControl): \{$[$ key: string$]$: any\} =$>$ \{
\par\ \ \ \ const forbidden = nameRe.test(control.value);
\par\ \ \ \ return forbidden ? \{'forbiddenName': \{value: control.value\}\}: null;
\par\ \ \};
\par \}
\}

\item 指令
\par @Directive(\{selector: '$[$forbiddenName$]$', providers: $[$\{provide: NG\_VALIDATORS, useExisting: ForbiddenValidatorDirective, multi: true\}$]$\})
\par export class ForbiddenValidatorDirective implements Validator \{  
\par\ \ @Input() forbiddenName: string;
\par\ \ validate(control: AbstractControl): \{$[$key: string$]$: any\} \{
\par\ \ \ \ return this.forbiddenName ? forbiddenNameValidator(new RegExp(this.forbiddenName, 'i'))(control): null;
\par\ \ \}\}


\end{enumerate}


推荐将数据输入,数据展示分开



\subsubsection{modal-driven}
ngOnInit(): void \{\\
this.heroForm = new FormGroup(\{
\par 'name': new FormControl(this.hero.name,
\par\ \  $[$Validators.required, Validators.minLength(4), forbiddenNameValidator(/bob/i) // $<$------ Here's how you pass in the custom validator. 
\par $]$),
\par 'alterEgo': new FormControl(this.hero.alterEgo),
\par 'power': new FormControl(this.hero.power, Validators.required)\\
\})\};

\subsubsection{注意}
\begin{enumerate}
\item 还没有写代码测试
\item 由于template-driven 基于directive,所以可能会导致渲染周期会不一致
\item 由于是基于modal的,必须等全部渲染完成才能和用户进行交互?


\end{enumerate}

\end{document}



