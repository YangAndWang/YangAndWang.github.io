\documentclass{article}

% 中文
\usepackage[UTF8]{ctex}
% 自动缩进
\usepackage{indentfirst}
% 支持跳转,标题的超链接
%\usepackage{hyperref}
\usepackage[colorlinks=true,linkcolor=blue]{hyperref}


% begin end环境
% flushleft:环境中的内容居左。
% flushright:环境中的内容居右。
% center:环境中的内容居中。
% itemize:无编号列表
% enumerate:有编号列表
% description:带描述列表
% quote:引用,使得整段缩进
% verse:诗歌专用,\\可以断行,两个空行的分段则生成一个空行。
% begin end 环境

\begin{document}
\title{angular学习}
\author{汪洋}
\maketitle

\newpage
\tableofcontents
\newpage


\section{Structural Directives}


\subsection{*ngIf}


$<$div *ngIf="hero"$>$\{\{hero.name\}\}$<$/div$>$

是一个语法糖,将被翻译成

\par $<$div template="ngIf hero"$>$\{\{hero.name\}\}$<$/div$>$

\par 接下来会生成

\par $<$ng-template $[$ngIf$]$="hero"$>$
\par\ \ $<$div$>$\{\{hero.name\}\}$<$/div$>$
\par $<$/ng-template$>$


\subsection{*ngFor}

$<$div *ngFor="let hero of heroes;let i = index;let odd=odd;trackBy: trackById" $[$class.odd$]$="odd"$>$
\par \{\{hero.name\}\}
\par $<$/div$>$
\par 将被翻译成\\
$<$div template="ngFor let hero of heroes; let i=index; let odd=odd; trackBy: trackById" $[$class.odd$]$="odd"$>$
\par  (\{\{i\}\}) \{\{hero.name\}\}\\
$<$/div$>$
\par 将生成\\
$<$ng-template ngFor let-hero $[$ngForOf$]$="heroes" let-i="index" let-odd="odd" $[$ngForTrackBy$]$="trackById"$>$
\par  $<$div $[$class.odd$]$="odd"$>$(\{\{i\}\}) \{\{hero.name\}\}$<$/div$>$\\
$<$/ng-template$>$


\par *ngFor 有几个内置的变量\ index odd \$implicit
\par 上面的hero由于没有值,所以使用\$implicit初始化了
\par 在*ngFor中声明的变量只存在于每个重复的实例中,在其他位置仍然可以声明相同名称的变量
\par 其他的几个内置的结构指令和*ngFor,*ngIf的展开方式一样

\subsection{ngSwitch}

ngSwitch 本省并不是一个结构指令,它只是给*ngSwitchCase提供一个值.


$<$div $[$ngSwitch$]$=hero?.emotion$>$  
\par $<$happy--hero *ngSwitchCase=happy $[$hero$]$=hero$>$$<$/happy--hero$>$
\par $<$sad--hero *ngSwitchCase=sad $[$hero$]$=hero$>$$<$/sad--hero$>$  
\par $<$confused--hero *ngSwitchCase=confused $[$hero$]$=hero$>$$<$/confused--hero$>$
\par $<$unknown--hero *ngSwitchDefault $[$hero$]$=hero$>$$<$/unknown--hero$>$
\par $<$/div$>$

将被翻译成

$<$div $[$ngSwitch$]$="hero?.emotion"$>$
\par $<$happy--hero template="ngSwitchCase 'happy'" $[$hero$]$="hero"$>$$<$/happy--hero$>$ 
\par $<$sad--hero template="ngSwitchCase 'sad'" $[$hero$]$="hero"$>$$<$/sad--hero$>$ 
\par $<$confused--hero template="ngSwitchCase 'confused'" $[$hero$]$="hero"$>$$<$/confused--hero$>$
\par $<$unknown--hero template="ngSwitchDefault" $[$hero$]$="hero"$>$$<$/unknown--hero$>$
\par $<$/div$>$

将生成

$<$div $[$ngSwitch$]$="hero?.emotion"$>$ 
\par $<$ng--template $[$ngSwitchCase$]$="'happy'"$>$ 
\par $<$happy--hero $[$hero$]$="hero"$>$$<$/happy--hero$>$ 
\par $<$/ng--template$>$  
\par $<$ng--template $[$ngSwitchCase$]$="'sad'"$>$ 
\par $<$sad--hero $[$hero$]$="hero"$>$$<$/sad--hero$>$
\par $<$/ng--template$>$
\par $<$ng--template $[$ngSwitchCase$]$="'confused'"$>$
\par $<$confused--hero $[$hero$]$="hero"$>$$<$/confused--hero$>$ 
\par $<$/ng--template $>$
\par $<$ng--template ngSwitchDefault$>$ 
\par $<$unknown--hero $[$hero$]$="hero"$>$$<$/unknown--hero$>$
\par $<$/ng--template$>$$<$/div$>$


\subsection{自定义结构指令}

\begin{enumerate}

	\item 构造函数有2个注入的参数
		\begin{enumerate}
			\item TemplateRef
			\item ViewContainerRef
		\end{enumerate}
	\item enumerate

\end{enumerate}

\subsection{注意}

angular 不允许同时在同一个元素中使用 多个结构指令

ng-template是angular内建的元素,如果它不与angular的指令配合使用的话,将被angular从Dom节点中移除

ng-container是angular内建的元素,但是与ng-template不同的是,它相当于一个语句块,本身不会显示出来,但是内部的其他符合\emph{结构指令}条件的元素会显示出来 


\section{Pipes}

管道使用$|$来标志

angular本身提供了全局的管道

DatePipe, UpperCasePipe, LowerCasePipe, CurrencyPipe, PercentPipe

\subsection{参数}
管道使用colon:来接受参数,如果有多个参数,则使用多个:
\par $|$date:'MM/dd/yy'
\par $|$slice:1:5

\subsection{Chaining pipes,管道链}

\subsection{自定义管道}
\begin{enumerate}
	\item 使用@Pipe,有一个name属性,以后可能会有pure属性
	\item 继承PipeTransform,transform方法至少一个参数value,然后是额外的参数
\end{enumerate}

\subsection{注意}
\begin{enumerate}
	\item 使用管道是昂贵的.每一次键盘事件,鼠标移动,时钟滴答,以及服务器响应后都会执行一次脏检查.
\end{enumerate}
\section{零碎}
\subsection{ngModal}
\begin{enumerate}
\item 必须引入FormsModule
\end{enumerate}
\end{document}



