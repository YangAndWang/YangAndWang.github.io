 
\documentclass{article} 

% 中文
% \usepackage[UTF8]{ctex}
% 自动缩进
\usepackage{indentfirst}
% 支持跳转,标题的超链接
% \usepackage{hyperref}
% 需要加上 使用的 pdf转化器
% http://blog.sina.com.cn/s/blog_7e3e55220100xgnt.html
\usepackage[dvipdfm,colorlinks=true,linkcolor=blue]{hyperref}


% begin end环境
% flushleft:环境中的内容居左。
% flushright:环境中的内容居右。
% center:环境中的内容居中。
% itemize:无编号列表
% enumerate:有编号列表
% description:带描述列表
% quote:引用,使得整段缩进
% verse:诗歌专用,\\可以断行,两个空行的分段则生成一个空行。
% begin end 环境

\begin{document}
\title{kotlin }
\author{wy}
\maketitle

\newpage
\tableofcontents
\newpage 

\section{keyword}
    \subsection{as}
         \begin{enumerate}
            \item as Type  convert variable to Other Type  
            \item as Name  alias otherName for import
         \end{enumerate}
    \subsection{as? Type}
         \begin{enumerate}
            \item as  convert variable to Other Type?  
         \end{enumerate}
    \subsection{by}
         \begin{enumerate}
             \item let the impl of interface delegate to anothor object
             \item delegate set/get value for var,or get value for val.\\
                 operator fun setValue(thisRef:Any$?$,property: KProperty$<$*$>$)\\
                 operator fun getValue(thisRef:Any$?$,property: KProperty$<$*$>$)   
         \end{enumerate}
    \subsection{delegate user agent}
    
    \subsection{dynamic} 
         \begin{enumerate}
             \item off kotlin type check for target JS
             \item use JS dynamic type
         \end{enumerate}

    \subsection{get,set}
         \begin{enumerate}
             \item get , var/val v : Type get() = v
             \item set , var v: Type set(value){ v = value;}, can't init
         \end{enumerate}
    \subsection{where}
         \begin{enumerate}
             \item genericity must have ability;
         \end{enumerate}
    \subsection{open,final}
         \begin{enumerate}
             \item final ,can't inherit
             \item open , opposite of Java's finalss
         \end{enumerate}
    \subsection{infix}
         \begin{enumerate}
             \item one method hava only one parameter
             \item marked as infix
             \item require last two,you can call it by | type fun param |
         \end{enumerate}
    \subsection{inner}
         \begin{enumerate}
             \item mark class as an inner class
         \end{enumerate}

    \subsection{private,protected,internal,public}
            \begin{enumerate}
                \item the default access control is public
                \item inherit access control.
            \end{enumerate}

            \subsubsection{in top-level in file}
                 \begin{enumerate}
                    \item public: can access every one
                    \item private: can access only in file
                    \item internal: can access in the same module
                    \item protected: illegal
                 \end{enumerate}
            \subsubsection{in class,interface}
                \begin{enumerate}
                    \item private: can access only in this class
                    \item protected: access int this class and it's subclass
                    \item internal: can access the internal member of which class it can see 
                    \item public: can access the member of which class it can see
                \end{enumerate}
            \subsection{in,out}
                \begin{enumerate}
                    \item same as java <? extends/super T>,
                    \item in:<in T>,Type may the super Type of T
                    \item out:<out T>,Type may the suber Type of T
                \end{enumerate}
            \subsection{inner}
                \begin{enumerate}
                    \item inner is modifer inner class,
                    \item inner class without inner is the static class
                \end{enumerate}
            \subsection{anonymous}
                \begin{enumerate}
                    \item the anonymous fun,is write with {}
                \end{enumerate}
\newpage
\section{doc}
    \subsection{fun call}
         \begin{enumerate}
            \item fun a(a:Int=1,b:Int){}, can be call by named param,i.e. a(b=2);
            \item named param should place after general param
         \end{enumerate}
    \subsection{fun override}
         \begin{enumerate}
             \item use keyword override, override fun b(){}
             \item the default param's value will be the base fun's default value
         \end{enumerate}
    \subsection{var val}
         \subsubsection{var}
            declare mutable variable
         \subsubsection{val}
            declare read-only variable
    \subsection{switch,when}
         

\end{document}